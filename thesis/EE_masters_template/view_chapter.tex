\documentclass[11pt, final]{ucthesis}

\bibliographystyle{IEEEtran}

% As the thesis gets bigger and bigger, you'll only want to look at a
% single chapter at a time. This is a template that prints just a single chapter
%
% Here, the COMMENT format might be particularly useful as the
% wide margins allow you or your committee to write long notes to
% yourself about how to improve things.
%
%%% \documentclass[11pt/12pt/10pt, final/draft/comment]{ucthesis}
%%% The first argument is the text size. The second argument changes
%%% The margins and the spacing.
%%% FINAL mode prints double spaced with figures and thesis margins.
%%% DRAFT mode prints single-spaced with 1" margins, with slightly
%%%   narrower margins for the figure captions so that they stand out.
%%% COMMENT mode prints an extra wide right margin and tiny margins
%%% everywhere else. This allows your committee plenty of room to add
%%% comments.
%%% Margins can be changed in the last lines of the ucthesis.cls
%%% with the geometry command.

\setlength{\parindent}{0.25in} \setlength{\parskip}{6pt}

\graphicspath{{introduction_figs/},{circles_figs/},{scaling_figs/},{bonus_capacity_figs/}}


% Degree Symbol
%\newcommand\degrees[1]{\ensuremath{^\cir#1}}
\newcommand\degrees{\ensuremath{^\circ}}
\newcommand{\tab}{\hspace{5mm}}


\begin{document}

\begin{dissertationText}
\renewcommand{\baselinestretch}{1.66}

% The title as you want it to appear at the top of the page and
% in the table of contents.
\chapter{Chapter Title}

% PITHY QUOTE
% In the draft version, I like to put a fun quote at the beginning of each
% chapter to keep myself entertained. I considered putting these in
% the final version as well, but I'll let you decide that for yourself.
% Note that the quote does affect spacing, if you're trying to look at
% how the final version of the chapter will appear in LaTeX...
\begin{quote}
Everyone has their little faults. Mine is in California.
\end{quote}
\begin{flushright}
-- Lex Luther
\end{flushright}

%%% The chapter is stored in a file called chapter2.tex
%%% The file should have no preamble -- the chapter body.
    \section{Spectrum allocation vs.~usage}
\label{sec:usage}

Looking at the NTIA's chart of these frequency allocations
(Figure~\ref{fig:spectrum_alloc}a), it appears that we are in danger
of running out of spectrum~\cite{NTIAspectrum}. However, allocation
is only half the story. Contrary to popular belief, actual
measurements (taken in downtown Berkeley, CA) show that most of the
allocated spectrum is vastly underutilized
(Figure~\ref{fig:spectrum_alloc}b)~\cite{BWRC2004, NewAm2005}.

\begin{figure}[htp]
\begin{center}
\subfigure[What does this graph mean?]{
  \epsfxsize=3in \epsfbox{spectrum2.eps}}
\subfigure[This picture makes no sense here.]{
  \epsfxsize=2.8in \epsfbox{usage.eps}}
\caption{\footnotesize{This picture is just here as a placeholder.
\label{fig:spectrum_alloc}}}
\end{center}
\end{figure}

blah blah blah...

Examining solely the 402 MHz allocated to broadcast TV
(Appendix~\ref{appen:commspec})...

\section{The policy debate}
\label{sec:policy}

Clearly, the spectrum is far from fully utilized. As a result, the
FCC's exclusive-use allocation policy is being increasingly viewed
as outdated.


\clearpage

\bibliography{IEEEabrv,cog_radio}


\end{dissertationText}
\end{document}
