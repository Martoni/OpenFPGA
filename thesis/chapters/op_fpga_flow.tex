\section{Design Flow Overview}
\label{sec:flow_over}

What is more significant is the tool flow that generates hardware and software. 
In this section, we describe all the tools we have used. Fig.~\ref{fig:flow_over} gives an overview of the tool flow.
 All the tools are open source with the exception of ASIC tool flow from Open Source FPGA RTL
 to ASIC. We describe the ASIC Flow in Appendix ~\ref{appen:tileGen}. \par

There are three inputs in the flow diagram, \textbf{fpga.scala}, \textbf{architecture.xml} and \textbf{user\_design.v}.
\textbf{fpga.scala} is a set of files that comprise as a generator for the FPGA. It is implemented in Chisel.
\textbf{architecture.xml} is the description of the FPGA. It is also input to the VTR flow. \cite{Rose:2012:VPA:2145694.2145708}
\textbf{user\_design.v} is the user's circuit to implement. \par

\begin{figure}[htp]
	\begin{center}
    \epsfysize=5in
		\epsfbox{flow_overview.eps}
		    \renewcommand{\captionfont}{\small}
				\caption{Design Flow Overview
				\label{fig:flow_over}}
	\end{center}
\end{figure}

\section{The Open Source FPGA Flow}
\label{sec:vtr_flow}

The core of the open source FPGA flow is the VTR Project. \cite{Rose:2012:VPA:2145694.2145708}
We extend the work to add FPGA RTL configuration generation capability and bit stream generation capability.
Fig.~\ref{fig:vtr_flow} illustrates the flow. All process inside the dashed box is from VTR Project.
In the next section, we describe the process flow inside the "FPGA Gen and Bit Gen" box. \par

\begin{figure}[htp]
	\begin{center}
    \epsfysize=6in
		\epsfbox{vtr_flow.eps}
		    \renewcommand{\captionfont}{\small}
				\caption{FPGA Flow
				\label{fig:vtr_flow}}
	\end{center}
\end{figure}

\section{FPGA Generation and Bitstream Generation Flow}
\label{sec:fg_bg_flow}

This part is a major work in the software flow. It is implemented in python. Fig.~\ref{fig:fp_bg_flow} illustrates
major steps inside the flow.

\begin{figure}[htp]
	\begin{center}
		\epsfxsize=5.0in
		\epsfbox{fg_bg_flow.eps}
		    \renewcommand{\captionfont}{\small}
				\caption{FPGA Generation and Bitstream Generation Flow
				\label{fig:fp_bg_flow}}
	\end{center}
\end{figure}