\section{Column Based FPGA Architecture}
\label{sec:column_fpga_arch}

Column based FPGA architecture has been adapted in commercial FPGA devices. \cite{xilinx:v4_arch}
It is characterized by each column of the FPGA has identical functional tiles by ignoring IO Tiles. (Fig.~\ref{fig:fpga_full_arch})
We choose the column based architecture as it offers simplicity in standard ASIC flow, is well supported in VTR and is 
relatively simple in generating bitstreams. Fig.~\ref{fig:fpga_full_arch}
gives a high level view of the architecture. It illustrates an FPGA that is 8 by 8 in tiles with IOs on the perimeter
and logic, storage and computing resources in the middle. \par

\begin{figure}[htp]
	\begin{center}
		\epsfxsize=4in 
		\epsfbox{fpga_full.eps}
		    \renewcommand{\captionfont}{\small}
				\caption{Full FPGA Arch
				\label{fig:fpga_full_arch}}
	\end{center}
\end{figure}

There are four different tiles in the architecture. They are IO Tiles, LUT Tiles, MEM Tiles, MUL Tiles.
 Section~\ref{sec:tile_funcs} provides detail descriptions on their functions. Commercial FPGAs
 have more advanced tiles to provide additional functionality such as A/D converters, differential IOs 
and memory controllers. Those circuits are mixed signal circuits and require special support in design 
and tools, so they are beyond the scope of this work. \par

\section{Tile Functions}
\label{sec:tile_funcs}
The four tiles have different functions with one common feature: they all have connections on their sides 
to make neighbor tiles connected. Non-neighbor tiles are indirectly connected via tiles among them. How they are
connected is part of the routing architecture which is described in Section~\ref{sec:routing_architecture}. We depict the difference of
their functions as follows. \par

\begin{description}
  \item[IO Tile] \hfill \\
  The IO Tile (Fig.~\ref{fig:inside_tiles}.(a)) terminates input routing tracks and initiate output routing tracks. It provides a gateway for the FPGA, so input and output
	to the FPGA are routed into the FPGA routing fabric. \par
	\item[LUT Tile] \hfill \\
  LUT Tile (Fig.~\ref{fig:inside_tiles}.(b)) stands for Look Up Table Tile. It is the foundation of FPGA as it contains Look Up Tables for implementing combinational
	functions and hard flip-flops for storage. \par
  \item[MEM Tile] \hfill \\
  MEM Tile (Fig.~\ref{fig:inside_tiles}.(c)) contains a dual ports configurable width memory accessible by the user. 
	It is a designed feature in the project but is not fully implemented and not yet supported.
	However, we have an evaluation work to test the performance of the tile. It demonstrates operating speed up to 600MHz for a 32KBits configurable widths memory.
	It is implemented with a 65nm process technology and all the implementation is in Verilog HDL. \par
	\item[MUL Tile] \hfill \\
	Implementing a generic multiplier is costly and has low performance using Look Up Tables (LUT). It also consumes precious routing resources as many LUTs are needed and
	interconnections have to be made. Hardware Multipliers are cost effective in implementing multiple functions. The MUL Tile (Fig.~\ref{fig:inside_tiles}.(d)) contains configurable and fracturable
	width hard multipliers. For example, a MUL Tile with a 36 by 36 multiplier can be configured into two 18 by 18 multipliers
	or four 9 by 9 multipliers. Implementing a similar function uses more than 100 LUT Tiles. It is a designed feature in the project but is not fully implemented and not yet supported.
	Our quick evaluation work just to test the performance of the tile reveals operating speed up to 700MHz. 
	It is implemented with a 65nm process technology and all the implementation is in Verilog HDL. \par
\end{description}

\begin{figure}[htp]
	\begin{center}
		\renewcommand{\captionfont}{\small}
		\subfigure[IO Tile]{
			\epsfxsize=2.0in \epsfbox{io_tile.eps}}
		\subfigure[LUT Tile]{
			\epsfxsize=2.8in \epsfbox{lut_tile.eps}}
		\subfigure[MEM Tile]{
			\epsfxsize=2.8in \epsfbox{mem_tile.eps}}
		\subfigure[MUL Tile]{
			\epsfxsize=2.8in \epsfbox{madd_tile.eps}}
		\caption{Structure of tiles}
		\label{fig:inside_tiles}
	\end{center}
\end{figure}

\section{Routing Architecture}
\label{sec:routing_architecture}

\subsection{High Level Overview}
\label{subsec:hlv_routing_arch}
On a high level view, there are horizontal and vertical wires in an FPGA. They are named \emph{X Track} and \emph{Y Track} respectively. A \emph{switch box (SW)} 
connects several tracks together. The architecture can be abstracted as a graph with no islands. 
That implies routability for any signal with one source and one destination. If the 
signal flows in only one way on a track, the track is called \emph{uni-directional}. If the signal can flow both ways on a track, the track is \emph{bi-directional}.
A track driven by a single signal is a \emph{single-driver} wire. The single signal is the output of a \emph{multiplexer (MUX)}. A track driven by several
signals is a \emph{multi-driver} wire. The sources are output of several MUXs and there are pass transistors to select one of the MUXs to be the source. \par

A routing track that spans more than one tile is called \emph{segmented track}. A segment~1 track connects two neighbor tiles. A segment~4 track connects five 
tiles either horizontally (a Segment~4 X Track) or vertically (a Segment~4 Y Track). Segmentation is especially useful for high performance global routing
as the number of MUXs a signal passes through is reduced. Fig.~\ref{fig:segementation} illustrate a segment of 4 wire in both X and Y direction. \par

\begin{figure}[htp]
	\begin{center}
		\epsfxsize=3in
		\epsfbox{routing_tracks.eps}
		    \renewcommand{\captionfont}{\small}
				\caption{Routing Tracks
				\label{fig:routing_track}}
	\end{center}
\end{figure}

\begin{figure}[htp]
	\begin{center}
		\renewcommand{\captionfont}{\small}
		
		\hspace*{\fill}
		\subfigure[Bidirectional]{
			\epsfxsize=1.5in \epsfbox{bidir.eps}} \hfill
		\subfigure[Unidirectional]{
			\epsfxsize=1.5in \epsfbox{unidir.eps}}
		\hspace*{\fill}	
			
		\subfigure[Single Driver]{
			\epsfxsize=2.5in \epsfbox{single_dri.eps}}
		\subfigure[Multi Driver]{
			\epsfxsize=2.75in \epsfbox{multi_dri.eps}}
		\caption{Routing Illustrated}
		\label{fig:r_illu}
	\end{center}
\end{figure}

\begin{figure}[htp]
	\begin{center}
		\epsfxsize=3.5in
		\epsfbox{segmentation.eps}
		    \renewcommand{\captionfont}{\small}
				\caption{Segmentation Illustrated.
				\label{fig:segementation}}
	\end{center}
\end{figure}

\subsection{Routing Track}
In this work we choose a single driver unidirectional routing architecture because of the following constraints and benefits:

\begin{enumerate}
  \item It is area efficient, energy efficient and enables higher clock speed. \cite{5377675}, \cite{1393249}
  \item VPR supports only single-driver routing. \cite{Luu:2011:VFC:2068716.2068718}
  \item Both bi-directional and multi-driver routing require the use of pass transistors which are not available in a standard cell ASIC flow. This is
	in contradiction with the objective to use only standard cell ASIC flow in the work. Also, pass transistors require special handling in 
	static timing analysis. 
\end{enumerate}

\subsection{Switch Box (SB)}
A switch box has switches that can connect two routing tracks. It is part of the inter-tile routing architecture. It has two variants: 
one is a simple design and is called \emph{planar topology}; the other is more
sophisticated but enables better connectivity and it called \emph{Wilton topology}. (Fig. \ref{fig:sb_illu})
Assuming routing tracks between two neighbor tiles are numbered
from 0 and up with track 0 being the left most track and bottom most track,
 the planar topology only connects tracks that have the same number; 
the Wilton topology connects tracks with the same or different track numbers so the topology is more flexible.
 In this work, we only use Wilton switch box topology. All the switch boxes are incorporated into tiles
 so the architecture diagram does not reflect their existence. \par

\begin{figure}[htp]
	\begin{center}
		\renewcommand{\captionfont}{\small}
		
		\hspace*{\fill}		
		\subfigure[Planar Topology]{
			\epsfxsize=1.75in \epsfbox{planar.eps}} \hfill
		\subfigure[Wilton Topology]{
			\epsfxsize=1.75in \epsfbox{wilton.eps}}	
		\hspace*{\fill}
		
		\caption{Switch Box Illustrated}
		\label{fig:sb_illu}
	\end{center}
\end{figure}

\subsection{Connection Box (CB)}
A connection box has switches that can connect a routing track to an \emph{input pin (IPIN)}. It is part the intra-tile routing architecture.
On the circuit level, it has a group of MUXs with routing tracks as input and IPINs as output. (Fig. \ref{fig:cb_illu})

\begin{figure}[htp]
	\begin{center}
		\renewcommand{\captionfont}{\small}
		
		\hspace*{\fill}		
		\subfigure[Connection Box]{
			\epsfxsize=2.0in \epsfbox{cb.eps}} \hfill
		\subfigure[Connection Circuit]{
			\epsfxsize=2.25in \epsfbox{cb_cir.eps}}
		\hspace*{\fill}		
		
		\caption{Connection Box Illustrated}
		\label{fig:cb_illu}
	\end{center}
\end{figure}

\section{Clocking Structure}
\label{sec:clocking}

All the edge-triggered elements in the FPGA are on the same clock domain. This is because 1) 
the open source FPGA tools do not support multi-clock domain circuit placement and timing analysis, 2) 
it adds complexity in physical implementation of the FPGA circuit 3) 
it significantly adds routing complexity and 4) the routing architecture needs
 to be modified for feeding clock signals into routing fabric. However, it is feasible to implement it as a future work. \par

In the physical implementation phase, the clock tree is synthesized, placed
 and routed with ASIC flow tools. \par

\section{Regularities in Tiling}
\label{sec:regular_tile}
Excluding IO Tiles, all tiles not on the boundary share the same configuration switch box and connection box. Therefore, we have
one tile generator for such type of tiles. Using a single generator enables us to take advantage of a faster ASIC flow as the 
placement and routing result inside many tiles is shared. Below is the actual code that implements a switch box and a
 connection box inside a non-boundary LUT Tile.\par

\begin{scala}
class sbcb extends Component {                                                                        
  val io = new Bundle {                                                                               
    val ipin_in = Bits(INPUT, VAR_NUM_IPIN_PER_TILE*VAR_IPIN_INPUT_WIDTH)                             
    val ipin_config = Bits(INPUT, VAR_NUM_IPIN_PER_TILE*VAR_IPIN_CONFIG_WIDTH)                        
                                                                                                      
    val chanxy_in = Bits(INPUT, TOTAL_CHANXY_IN)                                                      
    val chanxy_config = Bits(INPUT, VAR_TOTAL_CHANXY_CONFIGS)                                         
                                                                                                      
    val ipin_out = Bits(OUTPUT, VAR_NUM_IPIN_PER_TILE)                                                
    val chanxy_out = Bits(OUTPUT, VAR_NUM_CHANXY_PER_TILE)                                            
  }                                                                                                   
                                                                                                      
  val ipin_in_break = new Array [Bits] (VAR_NUM_IPIN_PER_TILE)                                        
  val ipin_config_break = new Array [Bits] (VAR_NUM_IPIN_PER_TILE)                                    
  val to_ipin_out = new Array [Bits] (VAR_NUM_IPIN_PER_TILE)                                          
  val chanxy_in_break = new Array [Bits] (VAR_NUM_CHANXY_PER_TILE)                                    
  val chanxy_config_break = new Array [Bits] (VAR_NUM_CHANXY_PER_TILE)                                
  val to_chanxy_out = new Array [Bits] (VAR_NUM_CHANXY_PER_TILE)                                      
                                                                                                      
  var i = 0;                                                                                          
                                                                                                      
  for ( i <- 0 until VAR_NUM_IPIN_PER_TILE) {                                                         
    ipin_in_break(i) = Bits (width = VAR_IPIN_INPUT_WIDTH)                                            
    ipin_in_break(i) := io.ipin_in((i+1)*VAR_IPIN_INPUT_WIDTH - 1, i*VAR_IPIN_INPUT_WIDTH)            
                                                                                                      
    ipin_config_break(i) = Bits (width = VAR_IPIN_CONFIG_WIDTH)                                       
    ipin_config_break(i) := io.ipin_config((i+1)*VAR_IPIN_CONFIG_WIDTH - 1, i*VAR_IPIN_CONFIG_WIDTH)  
                                                                                                      
    to_ipin_out(i) = Bits (width = 1)                                                                 
    to_ipin_out(i) := ipin_in_break(i)(ipin_config_break(i))                                          
  }                                                                                                   
                                                                                                      
  for ( i <- 0 until VAR_NUM_CHANXY_PER_TILE )                                                        
  {                                                                                                   
    var chanxy_in_start_bit = VAR_CHANXY_INPUT_START_ARR(i)                                         
    var chanxy_in_end_bit = VAR_CHANXY_INPUT_START_ARR(i) + VAR_CHANXY_INPUT_WIDTH_ARR(i) - 1       
                                                                                                      
    var chanxy_config_start_bit = VAR_CHANXY_CONFIG_START_ARR(i)                                    
    var chanxy_config_end_bit = VAR_CHANXY_CONFIG_START_ARR(i) + VAR_CHANXY_CONFIG_WIDTH_ARR(i) - 1 
                                                                                                      
    chanxy_in_break(i) = Bits (width =  VAR_CHANXY_INPUT_WIDTH_ARR(i))                              
    chanxy_in_break(i) := io.chanxy_in(chanxy_in_end_bit, chanxy_in_start_bit)                      
                                                                                                      
    chanxy_config_break(i) = Bits (width = VAR_CHANXY_CONFIG_WIDTH_ARR(i))                          
    chanxy_config_break(i) := io.chanxy_config(chanxy_config_end_bit, chanxy_config_start_bit)      
                                                                                                      
    to_chanxy_out(i) = Bits (width = 1)                                                             
    to_chanxy_out(i) := (chanxy_in_break(i))(chanxy_config_break(i))                                
  }                                                                                                   
                                                                                                      
  io.ipin_out := convert_Array_to_Bits (to_ipin_out, VAR_NUM_IPIN_PER_TILE)                           
  io.chanxy_out := convert_Array_to_Bits (to_chanxy_out, VAR_NUM_CHANXY_PER_TILE)                     
}                                                                                                     
\end{scala}

Some boundary tiles are unique so there is a more evolved generator for such tiles. Below is the actual code that implements a switch box and a
 connection box inside a boundary LUT Tile. \par

\begin{scala}
class sbcb_sp (param: lut_Param) extends Component {                                                             
  val io = new Bundle {                                                                                          
    val ipin_in = Bits(INPUT, VAR_NUM_IPIN_PER_TILE*VAR_IPIN_INPUT_WIDTH)                                        
    val ipin_config = Bits(INPUT, VAR_NUM_IPIN_PER_TILE*VAR_IPIN_CONFIG_WIDTH)                                   
                                                                                                                 
    val chanxy_in = Bits(INPUT, param.num_chanxy_in)                                                             
    val chanxy_config = Bits(INPUT, param.num_chanxy_configs)                                                    
                                                                                                                 
    val ipin_out = Bits(OUTPUT, VAR_NUM_IPIN_PER_TILE)                                                           
    val chanxy_out = Bits(OUTPUT, param.num_chanxy_out)                                                          
  }                                                                                                              
                                                                                                                 
  val ipin_in_break = new Array [Bits] (VAR_NUM_IPIN_PER_TILE)                                                   
  val ipin_config_break = new Array [Bits] (VAR_NUM_IPIN_PER_TILE)                                               
  val to_ipin_out = new Array [Bits] (VAR_NUM_IPIN_PER_TILE)                                                     
  val chanxy_in_break = new Array [Bits] (param.num_chanxy_out)                                                  
  val chanxy_config_break = new Array [Bits] (param.num_chanxy_configs)                                          
  val to_chanxy_out = new Array[Bits] (param.num_chanxy_out)                                                     
                                                                                                                 
  var i = 0;                                                                                                     
                                                                                                                 
  for ( i <- 0 until VAR_NUM_IPIN_PER_TILE) {                                                                    
    ipin_in_break(i) = Bits (width = VAR_IPIN_INPUT_WIDTH)                                                       
    ipin_in_break(i) := io.ipin_in((i+1)*VAR_IPIN_INPUT_WIDTH - 1, i*VAR_IPIN_INPUT_WIDTH)                       
                                                                                                                 
    ipin_config_break(i) = Bits (width = VAR_IPIN_CONFIG_WIDTH)                                                  
    ipin_config_break(i) := io.ipin_config((i+1)*VAR_IPIN_CONFIG_WIDTH - 1, i*VAR_IPIN_CONFIG_WIDTH)             
                                                                                                                 
    to_ipin_out(i) = Bits (width = 1)                                                                            
    to_ipin_out(i) := ipin_in_break(i)(ipin_config_break(i))                                                     
  }                                                                                                              
                                                                                                                 
  for ( i <- 0 until param.num_chanxy_out )                                                                      
  {                                                                                                              
      if ( param.chanxy_config_width_list(i) == 0 )                                                              
      {                                                                                                          
          to_chanxy_out(i) = Bits (width = 1)                                                                    
          to_chanxy_out(i) := io.chanxy_in ( param.chanxy_out_index_start_list(i) )                              
      }                                                                                                          
      else                                                                                                       
      {                                                                                                          
          var chanxy_in_start_bit = param.chanxy_out_index_start_list(i)                                         
          var chanxy_in_end_bit = param.chanxy_out_index_start_list(i) + param.chanxy_out_input_width_list(i) - 1
                                                                                                                 
          var chanxy_config_start_bit = param.chanxy_config_start_list(i)                                        
          var chanxy_config_end_bit = param.chanxy_config_start_list(i) + param.chanxy_config_width_list(i) - 1  
                                                                                                                 
          chanxy_in_break(i) = Bits (width = param.chanxy_out_input_width_list(i))                               
          chanxy_in_break(i) := io.chanxy_in(chanxy_in_end_bit, chanxy_in_start_bit)                             
                                                                                                                 
          chanxy_config_break(i) = Bits (width = param.chanxy_config_width_list(i))                              
          chanxy_config_break(i) := io.chanxy_config(chanxy_config_end_bit, chanxy_config_start_bit)             
                                                                                                                 
          to_chanxy_out(i) = Bits (width = 1)                                                                    
          to_chanxy_out(i) := (chanxy_in_break(i))(chanxy_config_break(i))                                       
      }                                                                                                          
  }                                                                                                              
                                                                                                                 
  io.ipin_out := convert_Array_to_Bits (to_ipin_out, VAR_NUM_IPIN_PER_TILE)                                      
  io.chanxy_out := convert_Array_to_Bits (to_chanxy_out, param.num_chanxy_out)                                   
}                                                                                                                
\end{scala}

\section{Architecture of Current Work}
\label{sec:lut_fpga_arch}
The MEM Tile and MUL Tile are implemented in hardware using Verilog HDL. However, because of the effort takes
in setting up an additional tool flow, we choose to leave it as part of the future work. Therefore, the fully implemented
architecture is like the one illustrated in Fig.~\ref{fig:fpga_lut_arch}. The are many parameters user can change and the
tool will handle all the rest of the work. Appendix~\ref{appen:op_param} lists all user accessible parameters and their
legal values. \par

\begin{figure}[htp]
	\begin{center}
		\epsfxsize=4in 
		\epsfbox{fpga_lut.eps}
		    \renewcommand{\captionfont}{\small}
				\caption{LUT FPGA Architecture
				\label{fig:fpga_lut_arch}}
	\end{center}
\end{figure}