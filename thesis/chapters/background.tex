\section{Research on Open Source FPGA Tools}
\label{sec:bkg_tools}

An FPGA has limited use without a toolflow which transforms a user RTL to a bitstream
just as an instruction set architecture has limited use without a compiler which transforms
a user program described in high level language to a piece of machine code. 
The VTR Project~\cite{Rose:2012:VPA:2145694.2145708} provides a toolflow from RTL written in Verilog
HDL to a placed and routed design. The work integrates ODIN II~\cite{Jamieson:2010:OIO:1827716.1827870},
ABC~\cite{Brayton:2010:AAI:2144310.2144317} and VPR~\cite{Luu:2011:VFC:2068716.2068718} into a flow.
They provide Verilog HDL elaboration, logic synthesis and technology mapping, packing, instances placing
and inter-instances routing respectively. The VTR Project provides us a foundation of the toolflow required.
All tools are open source and are publicly available. \par

\section{Research on Open Source FPGA Hardware}
\label{sec:bkg_hdw}

FPGA design and implementation are a time consuming process. It has
been estimated that the layout of an FPGA tile can take from 50 to 200 
man years to complete. \cite{Padalia:2003:ATP:611817.611842} Using a standard
cell flow in this project saves user effort. Some works have been devoted
 in studying the results of a standard cell flow based approach
in generating FPGA layouts and have demonstrated promising results. 
\cite{Neumann:2008:DFE:1403375.1403391}, \cite{Chaudhuri:2008:RRF:1391469.1391500}
One recent work even demonstrates that with a 45nm process technology, reduction 
in area and improvement in yield can be achieved at the same time
using a standard cell flow. \cite{Chen:2011:RFR:1950413.1950484} \par

One project that attempted to generate full FPGA layout with bitstream generation presented
 by Kuon et al from the University of Toronto \cite{Kuon:2005:DLV:1046192.1046220} is over seven years old
and only minor extensions are publicly available to this point. However the results from the work are promising. \par

With results from previous work, we believe it is totally feasible
and cost effective to implement an FPGA in RTL and 
to use standard ASIC tools doing the layout generation. \par

\section{The Chisel Language}
\label{sec:bkg_chisel}
Verilog HDL has been one of the de Facto languages in describing an RTL. However,
its support in parametrization and speed in the simulation are two big drawbacks of
the language. \par

In the Archipelago project, we use Chisel, an RTL generation language developed
at University of California Berkeley, to implement all the components in an FPGA.
\cite{Bachrach:2012:CCH:2228488.2228584}
Chisel stands for Constructing Hardware in Scala Embedded Language.
The language offers concise description of an RTL with versatile support on parametrization
and offers fast C++ cycle by cycle simulation. 
The top level connections are generated in Verilog HDL because Chisel cannot generate a circuit
with combinational paths. An FPGA must have combinational paths to have functionality and routability. It is
also beyond Chisel capability to perform large scale simulation as the C++ simulator flattens
a design. The flattened simulator can also be too large to fit in most host computers' caches. This 
 results very slow simulation speed. \par