\section{Look Up Table (LUT)}
\label{sec:lut}

A Look Up Table (LUT) implements combinational functions. The function
is programmed with configuration bits. It has two input signals, config\_in and config\_sel.
It has one output signal lut\_out. The width of the config\_sel has to be equal or more than two. Config\_in's 
width is two to the power of the width of config\_sel. For example, a 4-input 1-output LUT 
can implement any combinational circuit with four variables and one output and has 16 Bits
wide config\_in. Fig. \ref{fig:lut_illu}.(a) illustrates a 4-input 1-output 
LUT that implements $out=(a \oplus b) \land (\lnot c \lor d)$. \par

A LUT with 4-input, 5-input and 6-input provides the best area-delay product. \cite{1281800}
A 6-input LUT can be splitted into two 5-input LUTs to implement two different 5-input functions provided the two
functions' inputs are same. This only adds an overhead of one MUX. This scheme is illustrated in Fig. \ref{fig:lut_illu}(c). \par

\begin{figure}[htp]
	\begin{center}
		\renewcommand{\captionfont}{\small}
		\subfigure[4-input 1-output LUT Function 1]{
			\epsfxsize=2.5in \epsfbox{4-input-lut-func-1.eps}}
		\subfigure[4-input 1-output LUT Function 2]{
			\epsfxsize=2.5in \epsfbox{4-input-lut-func-2.eps}}
		
		\vspace{0.5cm}
			
		\subfigure[6-input fractured LUT]{
			\epsfxsize=3.5in \epsfbox{lut_frac.eps}}
		\caption{LUT Illustrated}
		\label{fig:lut_illu}
	\end{center}
\end{figure}

\section{Basic Logic Element (BLE)}
\label{sec:ble}

A pure combinational circuit is not as interesting as a computing machine with storage. A Flip Flop (FF) is a
clock edge triggered storage element. In our design, each BLE has a LUT and a FF. The FF can be bypassed, so
the output from the LUT is connected directly to the output of the BLE. (Fig.~\ref{fig:ble_illu}) Below is the code that implements
a BLE with a 6-input LUT and one FF. \par

\clearpage

\begin{scala}
class lut6 extends Component {                     
  val io = new Bundle {                            
    val lut_in = Bits(INPUT, 6)                    
    val lut_out = Bits(OUTPUT, 1)                  
    val lut_configs = Bits(INPUT, 64)              
    val mux_configs = Bits(INPUT, 1)               
    val ff_en = Bool(INPUT)                        
  }                                                
                                                   
  val lut6_o = io.lut_configs(io.lut_in)           
                                                   
  // declare one posedge flip-flops                
  val ff1 = Reg(resetVal = Bits("b0", 1))          
                                                   
  // ff_en is disabled while programming           
  when (io.ff_en) {                                
    ff1 := lut6_o                                  
  }                                                
                                                   
  io.lut_out := Cat(ff1, lut6_o)(io.mux_configs(0))
}                                                  
\end{scala}

The following code implements a BLE with a fracturable 6-input LUT and two FFs. It is not yet supported
in the tool flow (Bitstream Generation Phase) but is shown to demonstrate our work is designed to be expandable
at the hardware level. \par

\begin{scala}
// This is a 6-input fracturable lut with two outputs and two FFs                              
class lut6s extends Component {                                                               
  val io = new Bundle {                                                                       
    val lut_in  = Bits(INPUT, 6)                                                              
    val lut_out = Bits(OUTPUT, 2)                                                             
    val lut_configs = Bits(INPUT, 64)                                                         
    val mux_configs = Bits(INPUT, 3)                                                          
    val ff_en = Bool(INPUT)                                                                   
  }                                                                                           
                                                                                              
  // assign lut5 outputs from lut_config and lut input                                        
  val lut5_o0 = io.lut_configs(Cat(Bits(0), io.lut_in(4, 0)))                                 
  val lut5_o1 = io.lut_configs(Cat(Bits(1), io.lut_in(4, 0)))                                 
                                                                                              
  // assign lut6 outputs                                                                      
  val lut6_o = Cat(lut5_o1, lut5_o0)(io.lut_in(5))                                            
                                                                                              
  // declare two posedge flip-flops                                                           
  val ff1 = Reg(resetVal = Bits("b0", 1))                                                     
  val ff2 = Reg(resetVal = Bits("b0", 1))                                                     
                                                                                              
  // mux to select between lut5o0 and lut6                                                    
  val ff1_in = Cat(lut6_o, lut5_o0)(io.mux_configs(0))                                        
                                                                                              
  // ff_en is disabled while programming                                                      
  when (io.ff_en) {                                                                           
    ff1 := ff1_in                                                                             
    ff2 := lut5_o1                                                                            
  }                                                                                           
                                                                                              
  io.lut_out := Cat(Cat(ff2, lut5_o1)(io.mux_configs(2)), Cat(ff1, ff1_in)(io.mux_configs(1)))
}           
\end{scala}

\begin{figure}[htp]
	\begin{center}
		\renewcommand{\captionfont}{\small}
		\subfigure[6-input fractured LUT]{
			\epsfxsize=2.8in \epsfbox{lut_one_out.eps}}
		\subfigure[6-input LUT]{
			\epsfxsize=2.8in \epsfbox{lut_two_out.eps}}
		\caption{BLE Illustrated}
		\label{fig:ble_illu}
	\end{center}
\end{figure}

\section{Configurable Logic Block (CLB)}
\label{sec:clb}

Several BLEs clustered into a block is a CLB. (Fig.~\ref{fig:clb}) Previous studies has shown
a CLB with three to ten BLEs has the best area-delay product. \cite{1281800} Input to the CLBs
is the output of a crossbar switch to be described in Section \ref{sec:xbar}. \par

%\begin{wrapfigure}{r}{0.4\textwidth}
\begin{figure}[tbp]
	\begin{center}
		\epsfxsize=0.38\textwidth
		\epsfbox{clb.eps}
		    \renewcommand{\captionfont}{\small}
				\caption{CLB Illustrated
				\label{fig:clb}}
	\end{center}
\end{figure}
%\end{wrapfigure}

\section{Crossbar Switch (XBAR)}
\label{sec:xbar}
The crossbar switch provides routing support inside a CLB. The input to the crossbar switch is output of the connection box
and output of the CLB. It is a fully connected routing paradigm that is implemented using a MUXs Tree for each output signal. The implementation
is clean as it is a very regular structure. \par

\section{Inside A LUT Tile}
\label{sec:pit-lut-tile}
A LUT Tile contains a CLB, an XBAR, a CB and a SB. It also contains a configuration memory for controlling all the components. Details
on the configuration architecture are documented in Chapter ~\ref{chap:reconfig}. Fig.\ref{fig:inside_tiles}.(b) illustrates all the components. \par

\section{Inside an IO Tile}
\label{sec:pit-io-tile}

The IO Tile is simple. In this work, we assume input and output of an IO Tile only originates inside a chip, so no special drivers need to be
in place. It has no CLB and no XBAR, but only a CB and a SB. Configuration memory are also necessary. Detail
on the configuration architecture is documented in Chapter ~\ref{chap:reconfig}. Fig.\ref{fig:inside_tiles}.(a) illustrates all the components. \par

