\section{Tile ASIC Performance}
\label{sec:tile_asic_perf}

We sweep a combination of design parameters listed in Table~\ref{table:tile_perf_params}.
In this part of the work, we only push through a tile but not the entire FPGA to 
reduce ASIC flow speed. It is implemented in a 65nm process technology and 
standard cell flow. We report and analyze critical path delay, configuration memory area,
total tile area and static power. The maximum absolute operating speed is the highest
operating speed of the FPGA. The real operating speed depends on circuits implemented
on FPGA. \par

\begin{table}[htpb]
		\begin{center}
				{\tabulinesep=1.2mm
				\begin{tabu}{ | p{1.2in} | p{3.3in} |}    \hline
				Parameters & Valid Value and Description \\ \hline\hline
				LUT Size & 4, 5, 6 \\ \hline
				CLB Size & 4, 5, 6, 7, 8, 9, 10 \\ \hline
				IPIN Width & 8, 12, 16 \\ \hline
				CHANXY Width & 8, 12, 16 \\ \hline
				CHANXY Num & 80, 120, 160 \\ \hline
				\end{tabu}}
				\caption{Tile Performance Parameters
				\label{table:tile_perf_params}}
		\end{center}
\end{table}

We present all of the data in tables in Appendix~\ref{appen:tile_perf} for user
reference, so they can select configurations based on their use cases. Following
subsections only present result on varying one variable and analysis on the 
data trend. \par

\clearpage

\subsection{Effect of LUT Size}
\label{subsec:asic_eff_lut}

As LUT Size increases, configuration area, total area and static power all increases.
The delay in Fig.~\ref{fig:perf-lut}.(a) varies but still displays a general trend of 
higher delay as LUT Size increases. We believe the variation is because variation
in ASIC Tool performance. Sometimes the tool gives better QoRs. \par

\begin{figure}[htp]
	\begin{center}
		\renewcommand{\captionfont}{\small}
		\subfigure[Delay]{
			\epsfxsize=2.9in \epsfbox{c_lut_d.eps}}
		\subfigure[Configuration Area]{
			\epsfxsize=2.9in \epsfbox{c_lut_ca.eps}}
		\subfigure[Total Area]{
			\epsfxsize=2.9in \epsfbox{c_lut_ta.eps}}
		\subfigure[Static Power]{
			\epsfxsize=2.9in \epsfbox{c_lut_sp.eps}}
		\caption{LUT SIZE Effect on Performance}
		\label{fig:perf-lut}
	\end{center}
\end{figure}

\clearpage

\subsection{Effect of CLB Size}
\label{subsec:asic_eff_clb}

As CLB Size increases, CLB delay, configuration area, total area and static power all increases.
That is within our expectation. \par

\begin{figure}[htp]
	\begin{center}
		\renewcommand{\captionfont}{\small}
		\subfigure[Delay]{
			\epsfxsize=2.9in \epsfbox{c_clb_d.eps}}
		\subfigure[Configuration Area]{
			\epsfxsize=2.9in \epsfbox{c_clb_ca.eps}}
		\subfigure[Total Area]{
			\epsfxsize=2.9in \epsfbox{c_clb_ta.eps}}
		\subfigure[Static Power]{
			\epsfxsize=2.9in \epsfbox{c_clb_sp.eps}}
		\caption{CLB SIZE Effect on Performance}
		\label{fig:perf-clb}
	\end{center}
\end{figure}

\clearpage

\subsection{Effect of IPIN Width}
\label{subsec:asic_eff_ipin_w}

General performance trend matches our expectation: increased size leads to increased delay, area and power.
We believe the slight variation is due to ASIC tool variation. \par

\begin{figure}[htp]
	\begin{center}
		\renewcommand{\captionfont}{\small}
		\subfigure[Delay]{
			\epsfxsize=2.9in \epsfbox{c_ipin_d.eps}}
		\subfigure[Configuration Area]{
			\epsfxsize=2.9in \epsfbox{c_ipin_ca.eps}}
		\subfigure[Total Area]{
			\epsfxsize=2.9in \epsfbox{c_ipin_ta.eps}}
		\subfigure[Static Power]{
			\epsfxsize=2.9in \epsfbox{c_ipin_sp.eps}}
		\caption{IPIN Width Effect on Performance}
		\label{fig:perf-ipinw}
	\end{center}
\end{figure}

\clearpage

\subsection{Effect of CHANXY Width}
\label{subsec:asic_eff_chanxy_w}

General performance trend matches our expectation: increased size leads to increased delay, area and power.
We believe the slight variation is due to ASIC tool variation. \par

\begin{figure}[htp]
	\begin{center}
		\renewcommand{\captionfont}{\small}
		\subfigure[Delay]{
			\epsfxsize=2.9in \epsfbox{c_chanw_d.eps}}
		\subfigure[Configuration Area]{
			\epsfxsize=2.9in \epsfbox{c_chanw_ca.eps}}
		\subfigure[Total Area]{
			\epsfxsize=2.9in \epsfbox{c_chanw_ta.eps}}
		\subfigure[Static Power]{
			\epsfxsize=2.9in \epsfbox{c_chanw_sp.eps}}
		\caption{CHAN Width Effect on Performance}
		\label{fig:perf-chanw}
	\end{center}
\end{figure}

\clearpage

\subsection{Effect of Track Width}
\label{subsec:asic_eff_track_w}

As track number increases, delay, area and power all increases. Variation in delay is caused by ASIC tool variation.

\begin{figure}[htp]
	\begin{center}
		\renewcommand{\captionfont}{\small}
		\subfigure[Delay]{
			\epsfxsize=2.9in \epsfbox{c_chann_d.eps}}
		\subfigure[Configuration Area]{
			\epsfxsize=2.9in \epsfbox{c_chann_ca.eps}}
		\subfigure[Total Area]{
			\epsfxsize=2.9in \epsfbox{c_chann_ta.eps}}
		\subfigure[Static Power]{
			\epsfxsize=2.9in \epsfbox{c_chann_sp.eps}}
		\caption{CHAN NUM Effect on Performance}
		\label{fig:perf-chann}
	\end{center}
\end{figure}

\section{User Circuit Performance on FPGA}
\label{sec:vpr_perf}

\begin{table}[htpb]
		\begin{center}
				{\footnotesize
				{\tabulinesep=1.2mm
				\begin{tabu}{ | p{0.7in} | p{0.55in} | p{0.55in} | p{0.5in} | p{0.6in} | p{0.8in} | p{0.95in} | } \hline
				FPGA Size & LUT Size & CLB Size & IPIN W & CHAN W & CHAN NUM  & Max Freq \\ \hline\hline
				5 by 5 & 6 & 10 & 16 & 16 & 160 & 416.138 MHz \\ \hline
				10 by 10 & 5 & 8 & 12 & 12 & 120 & 377.845 MHz \\ \hline
				25 by 25 & 4 & 6 & 8 & 8 & 80 & 339.451 MHz \\ \hline
				\end{tabu}}}
				\caption{64 Bit Counter Performance
				\label{table:user_cir_perf}}
		\end{center}
\end{table}

We feed VPR with three sets of parameters and a 64 Bit counter to test
the performance of the FPGA. Table~\ref{table:user_cir_perf} lists the maximum operating
speed on three FPGAs we tested. As the goal of the project is not to aim the best performance, 
we report the performance number only for demonstrating the operating speed is good enough 
for real applications and presenting user some references so they can make their own decisions
in selecting architectural parameters. \par



