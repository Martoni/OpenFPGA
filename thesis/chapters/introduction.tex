\section{FPGA's Applications and Research}
\label{sec:fpga_app_res}

Field Programmable Gate Arrays (FPGAs) are effective in satisfying different computing
 requirements such as high performance computing~\cite{Saldana:2010:MPM:1862648.1862652}, 
digital signal processing~\cite{Chang:2005:DAR:1195478} and embedded computing~\cite{xilinx:embedded}. 
 Academic research efforts cover architecture optimization, design space exploration,
circuit level design and new CAD algorithms. \par

The Versatile Packing, Placement and Routing tool (VPR) developed at the University of 
Toronto~\cite{Betz:1997:VNP:647924.738755} has enabled a wide variety of researches
from novel FPGA architectures to new CAD algorithms. Following the 2011 FPGA Conference Evening
 Panel on "Should the academic community launch an open-source FPGA device and tools 
effort?"~\cite{Wawrzynek:2011:ACL:1950413.1950417} and the release of the Verilog to Routing (VTR)
 Project developed and integrated by the University of Toronto ~\cite{Rose:2012:VPA:2145694.2145708}, 
we believe it is worth the effort to design and implement an open source FPGA hardware with toolflow support. 
We name the project "Archipelago". An archipelago is composed of many islands; similarly, 
an FPGA is composed by many tiles. \par

\section{Goals and Design Decisions}
\label{sec:goals}
We designed the Archipelago architecture with two goals. First, it explores the quality of the physical
 implementation result that is produced by a standard ASIC design flow in a modern
ASIC process. Second, it enables other people to use and extend the project at will. 
The goal, however, is not to build the best performing FPGA in the world. With those 
goals in mind, we made several design decisions. \par

\begin{enumerate}
	\item To minimize human intervention, the tool flow is streamlined and scripted.
	\item The design is customizable with parameters. Parameters are centrally located in
	the architectural description file for VPR. Only a subset of parameters inside the
	architectural description file for VPR are designed into the Archipelago architecture.
	\item FPGA RTLs are fully synthesizable with a standard ASIC flow with no hand layout required.
	Users can perform performance evaluations of the physical implementation rather than deriving performance
	numbers with technology scaling.
	\item Users can treat this work as an FPGA generator and use it in their own circuits.
	\item Not all designed features are fully implemented and supported at this point, namely the Memory Tile
	and Multiplier Tile. \par
\end{enumerate}

\section{Report Structure}
\label{sec:rpt_struc}
This report is organized as follows. This chapter introduces the motivations, goals and design decisions
for the Archipelago project. Chapter~\ref{chap:background} provides a review on past work and related
development. Chapter~\ref{chap:arch} gives a top-down review of the Archipelago open source FPGA architecture. 
Chapter~\ref{chap:mod} documents a bottom-up description of modules at the RTL level. Chapter~\ref{chap:reconfig}
 describes the reconfiguration architecture and techniques used. Chapter~\ref{chap:flow} explains the supporting
 FPGA software with flow diagrams. Chapter~\ref{chap:testing} discusses functional test cases and methodologies.
Chapter~\ref{chap:performance} presents critical path delay, area and static
power consumption data and plots for different design parameters. 
Chapter~\ref{chap:conclusion} concludes the achievements in the work 
and suggests future directions of the project. \par