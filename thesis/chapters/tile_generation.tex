\section{ASIC Flow Overview}
\label{sec:asic_flow}

Input to a ASIC flow is RTL, in this case, in Verilog HDL. Constraints set Quality of Result (QoR) targets. Output of an ASIC flow is layout. (Fig.~\ref{fig:asic_flow_overview}) \par

\begin{figure}[htp]
	\begin{center}
		\epsfxsize=3.5in
		\epsfbox{asic_flow.eps}
		    \renewcommand{\captionfont}{\small}
				\caption{ASIC Flow Overview
				\label{fig:asic_flow_overview}}
	\end{center}
\end{figure}

\section{MIM Flow Using Synopsys IC Compiler}
\label{sec:mim_flow}

The Multiple Instantiated Module (MIM) is a flow that takes advantage of modules with same RTL and provides faster placement and routing and preserves QoR. Placing a large
design can take several days. The MIM flow requires a large design that contains many identical modules. In the MIM flow, the tool first performs global planning and places
all the identical modules. It then performs local placement and routing on only one of the module within the identical module group so the placement and routing step is only
executed once and available for all. This reduces physical implementation time. \par