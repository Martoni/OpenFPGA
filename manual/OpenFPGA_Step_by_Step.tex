\documentclass[10pt]{article}
\setlength\textwidth{6.875in}
\setlength\textheight{8.875in}
% set both margins to 2.5 pc
\setlength{\oddsidemargin}{-0.1875in}% 1 - (8.5 - 6.875)/2
\setlength{\evensidemargin}{-0.1875in}
\setlength{\marginparwidth}{0pc}
\setlength{\marginparsep}{0pc}
\setlength{\topmargin}{0in}
\setlength{\headheight}{0pt}
\setlength{\headsep}{0pt}
\setlength{\footskip}{37pt}%
\setlength{\parindent}{1pc}
\newcommand{\myMargin}{1.00in}
\usepackage[top=\myMargin, left=\myMargin, right=\myMargin, bottom=\myMargin, nohead]{geometry}
\usepackage{epsfig,graphicx}
\usepackage{palatino}
\usepackage{fancybox}
\usepackage{listings}
\lstset{aboveskip=10pt,belowskip=-5pt}
\usepackage{color}
\usepackage{hyperref}

\definecolor{mygreen}{rgb}{0,0.6,0}
\definecolor{mygray}{rgb}{0.5,0.5,0.5}
\definecolor{mymauve}{rgb}{0.58,0,0.82}

\lstset{ %
  backgroundcolor=\color{white},   % choose the background color; you must add \usepackage{color} or \usepackage{xcolor}
  basicstyle=\footnotesize,        % the size of the fonts that are used for the code
  breakatwhitespace=false,         % sets if automatic breaks should only happen at whitespace
  breaklines=true,                 % sets automatic line breaking
  captionpos=b,                    % sets the caption-position to bottom
  commentstyle=\color{mygreen},    % comment style
  deletekeywords={...},            % if you want to delete keywords from the given language
  escapeinside={\%*}{*)},          % if you want to add LaTeX within your code
  extendedchars=true,              % lets you use non-ASCII characters; for 8-bits encodings only, does not work with UTF-8
  frame=single,                    % adds a frame around the code
  keepspaces=true,                 % keeps spaces in text, useful for keeping indentation of code (possibly needs columns=flexible)
  keywordstyle=\color{blue},       % keyword style
  language=csh,                    % the language of the code
  morekeywords={*,...},            % if you want to add more keywords to the set
  numbers=none,                    % where to put the line-numbers; possible values are (none, left, right)
  rulecolor=\color{black},         % if not set, the frame-color may be changed on line-breaks within not-black text (e.g. comments (green here))
  showspaces=false,                % show spaces everywhere adding particular underscores; it overrides 'showstringspaces'
  showstringspaces=false,          % underline spaces within strings only
  showtabs=false,                  % show tabs within strings adding particular underscores
  stringstyle=\color{mymauve},     % string literal style
  tabsize=4,                       % sets default tabsize to 2 spaces
  title=\lstname                   % show the filename of files included with \lstinputlisting; also try caption instead of title
}

\newenvironment{commentary}
{ \vspace{-0.1in}
  \begin{quotation}
  \noindent
  \small \em
  \rule{\linewidth}{1pt}\\
}
{
  \end{quotation}
}

\title{OpenFPGA Step by Step Guide}
\author{Hao Jun Liu\\
EECS Department, UC Berkeley\\
{\tt  hjliu@eecs.berkeley.edu}
}
\date{\today}

\newenvironment{example}{\VerbatimEnvironment\begin{footnotesize}\begin{Verbatim}}{\end{Verbatim}\end{footnotesize}}
\newcommand{\kode}[1]{\begin{footnotesize}{\tt #1}\end{footnotesize}}

\def\code#1{{\small\tt #1}}

\def\note#1{\noindent{\bf [Note: #1]}}
%\def\note#1{}

\lstset{frame=, basicstyle={\footnotesize\ttfamily}}

\lstset{frame=}

\begin{document}
\maketitle{}

\section{Introduction}

This is a step by step guide on compiling and running the OpenFPGA tool flow. If there is any question, please contact
the author via email.

\section{Checking Out OpenFPGA}
\begin{lstlisting}[language=csh]
    svn checkout file:///users/hjliu/ProjSVN/open-fpga-complete
\end{lstlisting}

\section{Building VTR}
\begin{lstlisting}[language=csh]
    cd open-fpga-complete/vtr
    make
\end{lstlisting}

\noindent
From this point on, we reference all directories relative to the project directory (open-fpga-complete).
\newline

\section{Create an FPGA Architecture}

In this section, we will create a new FPGA architecture that is different from those that come with the project.
The FPGA we are going to create is an 8 by 8 tile, 6-input LUT, 4-LUT clustered CLB. It has 96 interconnects 
in each direction. Let's name the fpga configuration vpr\_6\_4\_10\_10\_96\_8\_8.xml. The naming convention is:


\begin{lstlisting}[language=csh]
    vpr_{LUT SIZE}_{CLB SIZE}_{IPIN W}_{CHANXY W}_{INTERCONNECTS}_x{X SIZE}_y{Y SIZE}.xml 
\end{lstlisting}  

\noindent    
The naming convection has to be kept as tools use it to derive variables. IPIN W and CHANXY W are informational only.
The tools do not need them. In order for the existing flow to work, INTERCONNECTS has to be an integer multiple of 8.

\begin{lstlisting}[language=csh]
    cd vtr/vtr_flow/arch/timing
    cp vpr_6_10_16_16_160_x5_y5.xml vpr_6_4_10_10_96_x8_y8.xml
\end{lstlisting}

\noindent
There are a few lines to be modified. Here are them:

\begin{lstlisting}[language=xml]
    L10: <layout width="8" height="8"/>
    L15: <area grid_logic_tile_area="8000"/>
    L85: <input name="I" num_pins="15" equivalent="true"/>
    L86: <output name="O" num_pins="4" equivalent="false"/>
    L90: <pb_type name="ble" num_pb="4">
    L134: <complete name="crossbar" input="clb.I ble[3:0].out" output="ble[3:0].in">
    L135: <delay_constant max="4.000000e-11" in_port="clb.I" out_port="ble[3:0].in" />
    L136: <delay_constant max="4.000000e-11" in_port="ble[3:0].out" out_port="ble[3:0].in" />
    L138: <complete name="clks" input="clb.clk" output="ble[3:0].clk">
    L140: <direct name="clbouts" input="ble[3:0].out" output="clb.O">
\end{lstlisting}

\section{Create a New Test Circuit}

We create a new circuit in this case, a decremental counter with a width of 16. Let name this dcounter.v.
We also need to create testbench and fpga testbench wrappers.

\begin{lstlisting}[language=csh]
    mkdir dcounter
    cp counter/counter_tb.v dcounter/dcounter_tb.v
    cp counter/counter.v dcounter/dcounter.v
    cp counter/fpga_wrapper.v dcounter/fpga_wrapper.v
    cp counter/run.sh dcounter/run.sh
    cp counter/counter_fpga_tb.v cp dcounter/dcounter_fpga_tb.v
    ln -s dcounter/dcounter.v dcounter.v
\end{lstlisting}

\noindent
Edit dcounter.v, dcounter\_tb.v and run.sh. Now simulate the dcounter with vsim.

\begin{lstlisting}[language=csh]
    vlib work
    vlog dcounter_tb.v dcounter.v
    ./run.sh > run.log
\end{lstlisting}

\noindent
Check run.log to make sure it is functional correct.

\noindent
Use the counter circuit as an example; edit dcounter\_fpga\_tb.v and fpga\_wrapper.v

\section{Run VTR Flow}

Edit the VTR config file and run test circuit.

\begin{lstlisting}[language=csh]
    cd vtr/vtr_flow/tasks/func_test_circuit/config
    vim config.txt
    cd vtr/vtr_flow/scripts
    ./open_fpga_run_vtr_task.pl func_test_circuit
\end{lstlisting}

\noindent
All circuits in different FPGA architectures should run OK.

\section{Create Test Dirs}

\begin{lstlisting}[language=csh]
    cd full-fpga-testing
    mkdir custom_ms
    cd custom_ms
    mkdir dcounter counter ff_en multi_consumer simple_comp single_inv single_inv_reg src wide_inv wide_inv_reg
    cd src
    ln -s ../../../archipelago/verilog/tools/cm_sp.v cm_sp.v
\end{lstlisting}

\section{Run OpenFPGA}

Create Directory Strcuture and run the flow.

\begin{lstlisting}[language=csh]
    cd vtr/vtr_flow/tools
    mkdir fpgaGen-vpr_6_4_10_10_96_x8_y8
    ../create_simlink.sh
    cp ../fpgaGen-vpr_6_10_16_16_160_x5_y5/*sh .
    rm -f temp.sh
    mv run_5_5_bs.sh run_8_8_bs.sh
    mv run_5_5.sh run_8_8.sh
\end{lstlisting}

\noindent
Edit shell scripts for the architecture. Teh run\_8\_8.sh is an important one. You need to put the right parameters
and directories as arguments to the Open FPGA tools.
The -{}-configs-en argument in the *\_bs.sh can be updated after you run the flow.
You can get the configuration depth after the flow. After the update, run the vtr flow first and then this flow again.

\begin{lstlisting}[language=csh]
    ./run_all.sh
    vim run_8_8_bs.sh
    ./run_all.sh
\end{lstlisting}

\section{Test Bitstreams on FPGA}

\begin{lstlisting}[language=csh]
    full-fpga-testing/custom_ms/dcounter
    vlib work; compile.sh
    run.sh
\end{lstlisting}

\end{document}
